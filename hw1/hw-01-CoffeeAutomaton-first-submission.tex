% Options for packages loaded elsewhere
\PassOptionsToPackage{unicode}{hyperref}
\PassOptionsToPackage{hyphens}{url}


\documentclass[
]{article}

%\usepackage{tcolorbox}
%\newtcolorbox{myquote}{colback=red!5!white, colframe=red!75!black}
%\newtcolorbox{myShaded}{colback=red!5!white, colframe=red!75!black}
% redefine the 'quote' environment to use this 'myquote' environment
%\renewenvironment{quote}{\begin{myquote}}{\end{myquote}}
%\newenvironment{Highlighting}{\begin{myquote}}{\end{myquote}}

\usepackage{lmodern}
\usepackage{amssymb,amsmath}


\usepackage{ifxetex,ifluatex}
\ifnum 0\ifxetex 1\fi\ifluatex 1\fi=0 % if pdftex
  \usepackage{mathpazo} % Palatino-like math fonts

  \usepackage[T1]{fontenc}
  \usepackage[utf8]{inputenc}
  \usepackage{textcomp} % provide euro and other symbols
\else % if luatex or xetex

% Load mathpazo as a math font
\usepackage{mathpazo}

% Load Pagella as a text font by specifying no-math to fontspec
%\usepackage[no-math]{fontspec}
%\setmainfont[Numbers=Proportional]{TeX Gyre Pagella}

  \usepackage{unicode-math}
  \defaultfontfeatures{Scale=MatchLowercase}
  \defaultfontfeatures[\rmfamily]{Ligatures=TeX,Scale=1}
  \setmainfont[Numbers = Proportional]{TeX Gyre Pagella}
  \setmonofont[]{Ubuntu Mono}
\fi
% Use upquote if available, for straight quotes in verbatim environments
\IfFileExists{upquote.sty}{\usepackage{upquote}}{}
\IfFileExists{microtype.sty}{% use microtype if available
  \usepackage[]{microtype}
  \UseMicrotypeSet[protrusion]{basicmath} % disable protrusion for tt fonts
}{}
\makeatletter
\@ifundefined{KOMAClassName}{% if non-KOMA class
  \IfFileExists{parskip.sty}{%
    \usepackage{parskip}
  }{% else
    \setlength{\parindent}{0pt}
    \setlength{\parskip}{6pt plus 2pt minus 1pt}}
}{% if KOMA class
  \KOMAoptions{parskip=half}}
\makeatother
\usepackage{xcolor}
\IfFileExists{xurl.sty}{\usepackage{xurl}}{} % add URL line breaks if available
\IfFileExists{bookmark.sty}{\usepackage{bookmark}}{\usepackage{hyperref}}
\hypersetup{
  pdftitle={Homework \#1},
  pdfauthor={Coffee Automaton},
  hidelinks,
  pdfcreator={LaTeX via pandoc}}
\urlstyle{same} % disable monospaced font for URLs
\usepackage[margin = 1in]{geometry}
\usepackage{color}
\usepackage{fancyvrb}
\newcommand{\VerbBar}{|}
\newcommand{\VERB}{\Verb[commandchars=\\\{\}]}
\DefineVerbatimEnvironment{Highlighting}{Verbatim}{commandchars=\\\{\}}
% Add ',fontsize=\small' for more characters per line
\usepackage{framed}
\definecolor{shadecolor}{RGB}{248,248,248}
\newenvironment{Shaded}{\begin{snugshade}}{\end{snugshade}}
\newcommand{\AlertTok}[1]{\textcolor[rgb]{0.94,0.16,0.16}{#1}}
\newcommand{\AnnotationTok}[1]{\textcolor[rgb]{0.56,0.35,0.01}{\textbf{\textit{#1}}}}
\newcommand{\AttributeTok}[1]{\textcolor[rgb]{0.77,0.63,0.00}{#1}}
\newcommand{\BaseNTok}[1]{\textcolor[rgb]{0.00,0.00,0.81}{#1}}
\newcommand{\BuiltInTok}[1]{#1}
\newcommand{\CharTok}[1]{\textcolor[rgb]{0.31,0.60,0.02}{#1}}
\newcommand{\CommentTok}[1]{\textcolor[rgb]{0.56,0.35,0.01}{\textit{#1}}}
\newcommand{\CommentVarTok}[1]{\textcolor[rgb]{0.56,0.35,0.01}{\textbf{\textit{#1}}}}
\newcommand{\ConstantTok}[1]{\textcolor[rgb]{0.00,0.00,0.00}{#1}}
\newcommand{\ControlFlowTok}[1]{\textcolor[rgb]{0.13,0.29,0.53}{\textbf{#1}}}
\newcommand{\DataTypeTok}[1]{\textcolor[rgb]{0.13,0.29,0.53}{#1}}
\newcommand{\DecValTok}[1]{\textcolor[rgb]{0.00,0.00,0.81}{#1}}
\newcommand{\DocumentationTok}[1]{\textcolor[rgb]{0.56,0.35,0.01}{\textbf{\textit{#1}}}}
\newcommand{\ErrorTok}[1]{\textcolor[rgb]{0.64,0.00,0.00}{\textbf{#1}}}
\newcommand{\ExtensionTok}[1]{#1}
\newcommand{\FloatTok}[1]{\textcolor[rgb]{0.00,0.00,0.81}{#1}}
\newcommand{\FunctionTok}[1]{\textcolor[rgb]{0.00,0.00,0.00}{#1}}
\newcommand{\ImportTok}[1]{#1}
\newcommand{\InformationTok}[1]{\textcolor[rgb]{0.56,0.35,0.01}{\textbf{\textit{#1}}}}
\newcommand{\KeywordTok}[1]{\textcolor[rgb]{0.13,0.29,0.53}{\textbf{#1}}}
\newcommand{\NormalTok}[1]{#1}
\newcommand{\OperatorTok}[1]{\textcolor[rgb]{0.81,0.36,0.00}{\textbf{#1}}}
\newcommand{\OtherTok}[1]{\textcolor[rgb]{0.56,0.35,0.01}{#1}}
\newcommand{\PreprocessorTok}[1]{\textcolor[rgb]{0.56,0.35,0.01}{\textit{#1}}}
\newcommand{\RegionMarkerTok}[1]{#1}
\newcommand{\SpecialCharTok}[1]{\textcolor[rgb]{0.00,0.00,0.00}{#1}}
\newcommand{\SpecialStringTok}[1]{\textcolor[rgb]{0.31,0.60,0.02}{#1}}
\newcommand{\StringTok}[1]{\textcolor[rgb]{0.31,0.60,0.02}{#1}}
\newcommand{\VariableTok}[1]{\textcolor[rgb]{0.00,0.00,0.00}{#1}}
\newcommand{\VerbatimStringTok}[1]{\textcolor[rgb]{0.31,0.60,0.02}{#1}}
\newcommand{\WarningTok}[1]{\textcolor[rgb]{0.56,0.35,0.01}{\textbf{\textit{#1}}}}
\setlength{\emergencystretch}{3em} % prevent overfull lines
\providecommand{\tightlist}{%
  \setlength{\itemsep}{0pt}\setlength{\parskip}{0pt}}
\setcounter{secnumdepth}{-\maxdimen} % remove section numbering

% text wrapping
\usepackage{fvextra}
\DefineVerbatimEnvironment{Highlighting}{Verbatim}{breaklines,commandchars=\\\{\}}
\renewenvironment{verbatim}{\begin{Shaded}}{\end{Shaded}}


\title{
  \vspace{2in}
  \textmd{\textbf{Homework \#1}}
  \normalsize\vspace{0.1in}\\
  \textmd{\textbf{CS 217 @ SJTU}}
  \normalsize\vspace{0.1in}\\
}

\author{Coffee Automaton}
\date{}

\begin{document}

\noindent
\large\textbf{Homework \#1}
\hfill
\textbf{Coffee Automaton} \\
\normalsize {\bf CS 217 @ SJTU} \hfill ACM Class, Zhiyuan College, SJTU\\
Prof.~{\bf Dominik Scheder} \hfill Due Date: September 23, 2019\\
  TA.~{\bf Tang Shuyang}
\hfill Submit Date: \today


\hypertarget{exercise-1}{%
\section{Exercise 1}\label{exercise-1}}

The integer division algorithm in school method is implemented in
\emph{\protect\hyperlink{appendix-1-euclidean-algorithm}{App.1}}.

The core code for integer division is as follows.

\begin{Shaded}
\begin{Highlighting}[]
\NormalTok{carry }\OperatorTok{=} \DecValTok{0}
\ControlFlowTok{for}\NormalTok{ i }\KeywordTok{in} \BuiltInTok{range}\NormalTok{(len_lhs }\OperatorTok{-}\NormalTok{ len_rhs, }\DecValTok{-1}\NormalTok{, }\DecValTok{-1}\NormalTok{):  }\CommentTok{# O(n - k)}
\NormalTok{    ok }\OperatorTok{=} \VariableTok{True}
    \ControlFlowTok{if}\NormalTok{ carry }\OperatorTok{==} \DecValTok{0}\NormalTok{:}
        \ControlFlowTok{for}\NormalTok{ j }\KeywordTok{in} \BuiltInTok{range}\NormalTok{(len_rhs }\OperatorTok{-} \DecValTok{1}\NormalTok{, }\DecValTok{-1}\NormalTok{, }\DecValTok{-1}\NormalTok{):  }\CommentTok{# O(k)}
            \ControlFlowTok{if}\NormalTok{ res_rem.b[i }\OperatorTok{+}\NormalTok{ j] }\OperatorTok{>}\NormalTok{ rhs.b[j]:}
\NormalTok{                ok }\OperatorTok{=} \VariableTok{True}
                \ControlFlowTok{break}
            \ControlFlowTok{elif}\NormalTok{ res_rem.b[i }\OperatorTok{+}\NormalTok{ j] }\OperatorTok{<}\NormalTok{ rhs.b[j]:}
\NormalTok{                ok }\OperatorTok{=} \VariableTok{False}
                \ControlFlowTok{break}
  
    \ControlFlowTok{if}\NormalTok{ ok:}
\NormalTok{        res_div.b[i] }\OperatorTok{=} \DecValTok{1}
        \ControlFlowTok{for}\NormalTok{ j }\KeywordTok{in} \BuiltInTok{range}\NormalTok{(}\DecValTok{0}\NormalTok{, len_rhs):  }\CommentTok{# O(k)}
\NormalTok{            res_rem.b[i }\OperatorTok{+}\NormalTok{ j] }\OperatorTok{-=}\NormalTok{ rhs.b[j]}
            \ControlFlowTok{if}\NormalTok{ res_rem.b[i }\OperatorTok{+}\NormalTok{ j] }\OperatorTok{<} \DecValTok{0}\NormalTok{:}
\NormalTok{                res_rem.b[i }\OperatorTok{+}\NormalTok{ j] }\OperatorTok{+=} \DecValTok{2}
\NormalTok{                res_rem.b[i }\OperatorTok{+}\NormalTok{ j }\OperatorTok{+} \DecValTok{1}\NormalTok{] }\OperatorTok{-=} \DecValTok{1}
    \ControlFlowTok{else}\NormalTok{:}
\NormalTok{        res_div.b[i] }\OperatorTok{=} \DecValTok{0}
\NormalTok{    carry }\OperatorTok{=}\NormalTok{ res_rem.b[i }\OperatorTok{+}\NormalTok{ len_rhs }\OperatorTok{-} \DecValTok{1}\NormalTok{]}
  
\end{Highlighting}
\end{Shaded}

The outer loop runs \(n-k\) times, and the inner loop runs \(k\) times.
Inside the loop body are bit operations. So the total algorithm runs in
\(O(k(n−k))\) operations.

\hypertarget{exercise-2}\NormalTok{ b }\CommentTok{# so a = bu + r}
        \ControlFlowTok{if}\NormalTok{ r }\OperatorTok{==} \DecValTok{0}\NormalTok{:}
            \ControlFlowTok{return}\NormalTok{ b}
\NormalTok{        s }\OperatorTok{=}\NormalTok{ b }\OperatorTok{%}\NormalTok{ r }\CommentTok{# so b = rv + s}
\NormalTok{        a }\OperatorTok{=}\NormalTok{ r}
\NormalTok{        b }\OperatorTok{=}\NormalTok{ s}
    \ControlFlowTok{return}\NormalTok{ a}
  
\end{Highlighting}
\end{Shaded}

It is obvious that \(r\le\lfloor\frac{a}{2}\rfloor\) and
\(s\le\lfloor\frac{b}{2}\rfloor\), so every cycle \(a\) and \(b\) become
at most half of the previous value they have.

Suppose \(a\ge b\), and let \(n\) and \(m\) denote the number of bits of
\(a\) and \(b\), we have \(n\ge m\).

Since the integer division makes \(O(m(n-m))\) bit operations.

The \(\gcd(a,b)\) makes

\[O(m(n-m)+\frac{1}{2^2}m(n-m)+\frac{1}{2^4}m(n-m)+\cdots)=O(m(n-m))=O(n^2)\]

operations.

\hypertarget{exercise-3}{%
\section{Exercise 3}\label{exercise-3}}

\begin{Shaded}
\begin{Highlighting}[]
\KeywordTok{def}\NormalTok{ combination_rec(n, k):}
    \ControlFlowTok{if}\NormalTok{ k }\OperatorTok{<} \DecValTok{0} \KeywordTok{or}\NormalTok{ k }\OperatorTok{>}\NormalTok{ n:}
        \ControlFlowTok{return} \DecValTok{0}
    \ControlFlowTok{elif}\NormalTok{ n }\OperatorTok{==} \DecValTok{0}\NormalTok{:}
        \ControlFlowTok{return} \DecValTok{1}
  
    \ControlFlowTok{return}\NormalTok{ combination_rec(n }\OperatorTok{-} \DecValTok{1}\NormalTok{, k) }\OperatorTok{+}\NormalTok{ combination_rec(n }\OperatorTok{-} \DecValTok{1}\NormalTok{, k }\OperatorTok{-} \DecValTok{1}\NormalTok{)}
  
  
\BuiltInTok{print}\NormalTok{(combination_rec(}\DecValTok{5}\NormalTok{, }\DecValTok{3}\NormalTok{))}
  
\end{Highlighting}
\end{Shaded}

\begin{Shaded}
\begin{Highlighting}[]
\NormalTok{10}
\end{Highlighting}
\end{Shaded}

The algorithm recurses \(O(nC_n^k)\) times, and the time complexity of
bitwise addition is \(O({\log_2 2^n})=O(n)\).

So the running time of the algorithm is \(O(n^2C_n^k)\).

It is not an efficient algorithm, because the time complexity of the
algorithm is exponential.

\hypertarget{exercise-4}{%
\section{Exercise 4}\label{exercise-4}}

The dynamic programming algorithm is implemented as follows.

\begin{Shaded}
\begin{Highlighting}[]
\KeywordTok{def}\NormalTok{ combination_dp(n, k):}
    \ControlFlowTok{if}\NormalTok{ k }\OperatorTok{<} \DecValTok{0} \KeywordTok{or}\NormalTok{ k }\OperatorTok{>}\NormalTok{ n:}
        \ControlFlowTok{return} \DecValTok{0}
    \ControlFlowTok{elif}\NormalTok{ n }\OperatorTok{==} \DecValTok{0}\NormalTok{:}
        \ControlFlowTok{return} \DecValTok{1}
  
    \CommentTok{"""}
\CommentTok{    C = [0] * (k + 1)}
\CommentTok{    for i in range(n + 1):}
\CommentTok{        C[0] = 1}
\CommentTok{        for j in range(min(i, k), 0, -1):}
\CommentTok{            C[j] = C[j] + C[j - 1]}
\CommentTok{    return C[k]}
\CommentTok{    """}
\NormalTok{    C }\OperatorTok{=}\NormalTok{ [}\VariableTok{None}\NormalTok{] }\OperatorTok{*}\NormalTok{ (n }\OperatorTok{+} \DecValTok{1}\NormalTok{)}
    \ControlFlowTok{for}\NormalTok{ i }\KeywordTok{in} \BuiltInTok{range}\NormalTok{(n }\OperatorTok{+} \DecValTok{1}\NormalTok{):}
\NormalTok{        C[i] }\OperatorTok{=}\NormalTok{ [}\DecValTok{0}\NormalTok{] }\OperatorTok{*}\NormalTok{ (k }\OperatorTok{+} \DecValTok{1}\NormalTok{)}
\NormalTok{        C[i][}\DecValTok{0}\NormalTok{] }\OperatorTok{=} \DecValTok{1}
        \ControlFlowTok{for}\NormalTok{ j }\KeywordTok{in} \BuiltInTok{range}\NormalTok{(}\DecValTok{1}\NormalTok{, }\BuiltInTok{min}\NormalTok{(i }\OperatorTok{+} \DecValTok{1}\NormalTok{, k }\OperatorTok{+} \DecValTok{1}\NormalTok{)):}
\NormalTok{            C[i][j] }\OperatorTok{=}\NormalTok{ C[i }\OperatorTok{-} \DecValTok{1}\NormalTok{][j] }\OperatorTok{+}\NormalTok{ C[i }\OperatorTok{-} \DecValTok{1}\NormalTok{][j }\OperatorTok{-} \DecValTok{1}\NormalTok{]}
    \ControlFlowTok{return}\NormalTok{ C[n][k]}
  
  
\BuiltInTok{print}\NormalTok{(combination_dp(}\DecValTok{5}\NormalTok{, }\DecValTok{3}\NormalTok{))}
  
\end{Highlighting}
\end{Shaded}

\begin{Shaded}
\begin{Highlighting}[]
\NormalTok{10}
\end{Highlighting}
\end{Shaded}

The outer loop runs \(n\) times, and the inner loop runs \(k\) times.
And the time complexity of bitwise addition is \(O({\log_2 2^n})=O(n)\).
So the running time of the program is \(O(n^2k)\).

It is an efficient algorithm, because the time complexity of the
algorithm is polynomial.

\hypertarget{exercise-5}{%
\section{Exercise 5}\label{exercise-5}}

The dynamic programming algorithm is implemented as follows.

\begin{Shaded}
\begin{Highlighting}[]
\KeywordTok{def}\NormalTok{ combination_dp_mod_2(n, k):}
    \ControlFlowTok{if}\NormalTok{ k }\OperatorTok{<} \DecValTok{0} \KeywordTok{or}\NormalTok{ k }\OperatorTok{>}\NormalTok{ n:}
        \ControlFlowTok{return} \DecValTok{0}
    \ControlFlowTok{elif}\NormalTok{ n }\OperatorTok{==} \DecValTok{0}\NormalTok{:}
        \ControlFlowTok{return} \DecValTok{1}
  
    \CommentTok{"""}
\CommentTok{    C = [0] * (k + 1)}
\CommentTok{    for i in range(n + 1):}
\CommentTok{        C[0] = 1}
\CommentTok{        for j in range(min(i, k), 0, -1):}
\CommentTok{            C[j] = (C[j] + C[j - 1]) % 2}
\CommentTok{    return C[k]}
\CommentTok{    """}
\NormalTok{    C }\OperatorTok{=}\NormalTok{ [}\VariableTok{None}\NormalTok{] }\OperatorTok{*}\NormalTok{ (n }\OperatorTok{+} \DecValTok{1}\NormalTok{)}
    \ControlFlowTok{for}\NormalTok{ i }\KeywordTok{in} \BuiltInTok{range}\NormalTok{(n }\OperatorTok{+} \DecValTok{1}\NormalTok{):}
\NormalTok{        C[i] }\OperatorTok{=}\NormalTok{ [}\DecValTok{0}\NormalTok{] }\OperatorTok{*}\NormalTok{ (k }\OperatorTok{+} \DecValTok{1}\NormalTok{)}
\NormalTok{        C[i][}\DecValTok{0}\NormalTok{] }\OperatorTok{=} \DecValTok{1}
        \ControlFlowTok{for}\NormalTok{ j }\KeywordTok{in} \BuiltInTok{range}\NormalTok{(}\DecValTok{1}\NormalTok{, }\BuiltInTok{min}\NormalTok{(i }\OperatorTok{+} \DecValTok{1}\NormalTok{, k }\OperatorTok{+} \DecValTok{1}\NormalTok{)):}
\NormalTok{            C[i][j] }\OperatorTok{=}\NormalTok{ (C[i }\OperatorTok{-} \DecValTok{1}\NormalTok{][j] }\OperatorTok{+}\NormalTok{ C[i }\OperatorTok{-} \DecValTok{1}\NormalTok{][j }\OperatorTok{-} \DecValTok{1}\NormalTok{]) }\OperatorTok{%} \DecValTok{2}
    \ControlFlowTok{return}\NormalTok{ C[n][k]}
  
  
\BuiltInTok{print}\NormalTok{(combination_dp_mod_2(}\DecValTok{5}\NormalTok{, }\DecValTok{3}\NormalTok{))}
  
\end{Highlighting}
\end{Shaded}

\begin{Shaded}
\begin{Highlighting}[]
\NormalTok{10}
\end{Highlighting}
\end{Shaded}

The outer loop runs \(n\) times, and the inner loop runs \(k\) times.
And the time complexity of bitwise addition modulo \(2\) is \(O(1)\). So
the running time of the program is \(O(nk)\).

It is an efficient algorithm, because the time complexity of the
algorithm is polynomial.

\hypertarget{exercise-6}{%
\section{Exercise 6}\label{exercise-6}}

A lasso cannot happen.

If we had some \(1\le i<j\) such that \(F'_i=F'_j\) and
\(F'_{i+1}=F'_{j+1}\), we would have
\(F'_{i-1}\equiv F'_{i+1}-F'_i\equiv F'_{j+1}-F'_j\equiv F'_{j-1}\pmod k\)
and \(F'_i=F'_j\), so \(i-1\) also satisfies the condition.

Therefore, the smallest \(i\) must be \(0\), and form a circle.

Here's the code for finding a fibonacci period:

\begin{Shaded}
\begin{Highlighting}[]
\KeywordTok{def}\NormalTok{ fibonacci_period_mod_k(k):}
\NormalTok{    period }\OperatorTok{=} \DecValTok{0}
\NormalTok{    f0, f1 }\OperatorTok{=} \DecValTok{0}\NormalTok{, }\DecValTok{1} \OperatorTok\NormalTok{ k}
  
        \ControlFlowTok{if}\NormalTok{ (f0, f1) }\OperatorTok{==}\NormalTok{ (}\DecValTok{0}\NormalTok{, }\DecValTok{1} \OperatorTok{%}\NormalTok{ k):}
            \ControlFlowTok{break}
  
    \ControlFlowTok{return}\NormalTok{ period}
  
  
\BuiltInTok{print}\NormalTok{(}\SpecialStringTok{f"period = }\SpecialCharTok{\{}\NormalTok{fibonacci_period_mod_k(}\DecValTok{3}\NormalTok{)}\SpecialCharTok{\}}\SpecialStringTok{"}\NormalTok{)}
  
\end{Highlighting}
\end{Shaded}

\begin{Shaded}
\begin{Highlighting}[]
\NormalTok{period = 8}
\end{Highlighting}
\end{Shaded}

\pagebreak

\hypertarget{appendix}{%
\section{Appendix}\label{appendix}}

\hypertarget{appendix-1-euclidean-algorithm}{%
\subsection{Appendix 1. Euclidean
algorithm}\label{appendix-1-euclidean-algorithm}}

\begin{Shaded}
\begin{Highlighting}[]
\KeywordTok{def}\NormalTok{ bit_len(n):}
    \CommentTok{"""}
\CommentTok{    Return bit length of n.}
\CommentTok{    Time complexity: O(|n|)}
\CommentTok{    """}
\NormalTok{    result }\OperatorTok{=} \DecValTok{0}
    \ControlFlowTok{while}\NormalTok{ n }\OperatorTok{>} \DecValTok{0}\NormalTok{:}
\NormalTok{        n }\OperatorTok{>>=} \DecValTok{1}
\NormalTok{        result }\OperatorTok{+=} \DecValTok{1}
  
    \ControlFlowTok{return}\NormalTok{ result}
  
  
\KeywordTok{class}\NormalTok{ Binary:}
    \KeywordTok{def} \FunctionTok{__init__}\NormalTok{(}\VariableTok{self}\NormalTok{, value}\OperatorTok{=}\VariableTok{None}\NormalTok{):}
        \ControlFlowTok{if} \BuiltInTok{isinstance}\NormalTok{(value, }\BuiltInTok{int}\NormalTok{):}
\NormalTok{            n_bits }\OperatorTok{=}\NormalTok{ bit_len(value)}
            \VariableTok{self}\NormalTok{.b }\OperatorTok{=}\NormalTok{ [}\DecValTok{0}\NormalTok{] }\OperatorTok{*}\NormalTok{ n_bits}
            \ControlFlowTok{for}\NormalTok{ i }\KeywordTok{in} \BuiltInTok{range}\NormalTok{(n_bits):}
                \VariableTok{self}\NormalTok{.b[i] }\OperatorTok{=}\NormalTok{ value }\OperatorTok{%} \DecValTok{2}
\NormalTok{                value }\OperatorTok{//=} \DecValTok{2}
        \ControlFlowTok{elif} \BuiltInTok{isinstance}\NormalTok{(value, }\BuiltInTok{list}\NormalTok{):}
            \VariableTok{self}\NormalTok{.b }\OperatorTok{=}\NormalTok{ value}
  
    \KeywordTok{def} \FunctionTok{__str__}\NormalTok{(}\VariableTok{self}\NormalTok{):}
\NormalTok{        integer }\OperatorTok{=} \DecValTok{0}
        \ControlFlowTok{for}\NormalTok{ num }\KeywordTok{in} \VariableTok{self}\NormalTok{.b[::}\OperatorTok{-}\DecValTok{1}\NormalTok{]:}
\NormalTok{            integer }\OperatorTok{=}\NormalTok{ integer }\OperatorTok{<<} \DecValTok{1} \OperatorTok{|}\NormalTok{ num}
        \ControlFlowTok{return} \BuiltInTok{str}\NormalTok{(integer)}
  
    \KeywordTok{def}\NormalTok{ copy(}\VariableTok{self}\NormalTok{):}
\NormalTok{        result }\OperatorTok{=}\NormalTok{ Binary()}
\NormalTok{        result.b }\OperatorTok{=} \VariableTok{self}\NormalTok{.b.copy()}
        \ControlFlowTok{return}\NormalTok{ result}
  
    \KeywordTok{def} \BuiltInTok{len}\NormalTok{(}\VariableTok{self}\NormalTok{):}
        \ControlFlowTok{return} \BuiltInTok{len}\NormalTok{(}\VariableTok{self}\NormalTok{.b)}
  
    \KeywordTok{def}\NormalTok{ add(}\VariableTok{self}\NormalTok{, rhs):}
        \ControlFlowTok{if} \KeywordTok{not} \BuiltInTok{isinstance}\NormalTok{(rhs, Binary):}
            \ControlFlowTok{raise} \PreprocessorTok{TypeError}\NormalTok{(}\StringTok{"rhs is not Binary"}\NormalTok{)}
  
\NormalTok{        len_lhs, len_rhs }\OperatorTok{=} \VariableTok{self}\NormalTok{.}\BuiltInTok{len}\NormalTok{(), rhs.}\BuiltInTok{len}\NormalTok{()}
  
\NormalTok{        result }\OperatorTok{=}\NormalTok{ []}
  
\NormalTok{        carry }\OperatorTok{=} \DecValTok{0}
  
        \ControlFlowTok{for}\NormalTok{ i }\KeywordTok{in} \BuiltInTok{range}\NormalTok{(}\BuiltInTok{max}\NormalTok{(len_lhs, len_rhs)):}
\NormalTok{            s }\OperatorTok{=} \DecValTok{0}
            \ControlFlowTok{if}\NormalTok{ i }\OperatorTok{<}\NormalTok{ len_lhs:}
\NormalTok{                s }\OperatorTok{+=} \VariableTok{self}\NormalTok{.b[i]}
            \ControlFlowTok{if}\NormalTok{ i }\OperatorTok{<}\NormalTok{ len_rhs:}
\NormalTok{                s }\OperatorTok{+=}\NormalTok{ rhs.b[i]}
\NormalTok{            result.append(s }\OperatorTok{+}\NormalTok{ carry)}
  
            \ControlFlowTok{if}\NormalTok{ result[}\OperatorTok{-}\DecValTok{1}\NormalTok{] }\OperatorTok{<} \DecValTok{2}\NormalTok{:}
\NormalTok{                carry }\OperatorTok{=} \DecValTok{0}
            \ControlFlowTok{else}\NormalTok{:}
\NormalTok{                carry }\OperatorTok{=} \DecValTok{1}
\NormalTok{                result[}\OperatorTok{-}\DecValTok{1}\NormalTok{] }\OperatorTok{-=} \DecValTok{2}
  
        \ControlFlowTok{if}\NormalTok{ carry }\OperatorTok{>} \DecValTok{0}\NormalTok{:}
\NormalTok{            result.append(carry)}
  
        \ControlFlowTok{return}\NormalTok{ Binary(result)}
  
    \KeywordTok{def}\NormalTok{ _div_rem(}\VariableTok{self}\NormalTok{, rhs):}
        \ControlFlowTok{if} \KeywordTok{not} \BuiltInTok{isinstance}\NormalTok{(rhs, Binary):}
            \ControlFlowTok{raise} \PreprocessorTok{TypeError}\NormalTok{(}\StringTok{"rhs is not Binary"}\NormalTok{)}
  
\NormalTok{        len_lhs, len_rhs }\OperatorTok{=} \VariableTok{self}\NormalTok{.}\BuiltInTok{len}\NormalTok{(), rhs.}\BuiltInTok{len}\NormalTok{()}
  
\NormalTok{        res_div, res_rem }\OperatorTok{=}\NormalTok{ Binary(}\DecValTok{0}\NormalTok{), }\VariableTok{self}\NormalTok{.copy()}
  
\NormalTok{        res_div.b }\OperatorTok{=}\NormalTok{ [}\DecValTok{0}\NormalTok{] }\OperatorTok{*}\NormalTok{ (len_lhs }\OperatorTok{-}\NormalTok{ len_rhs }\OperatorTok{+} \DecValTok{1}\NormalTok{)}
  
\NormalTok{        carry }\OperatorTok{=} \DecValTok{0}
        \ControlFlowTok{for}\NormalTok{ i }\KeywordTok{in} \BuiltInTok{range}\NormalTok{(len_lhs }\OperatorTok{-}\NormalTok{ len_rhs, }\DecValTok{-1}\NormalTok{, }\DecValTok{-1}\NormalTok{):  }\CommentTok{# O(n - k)}
\NormalTok{            ok }\OperatorTok{=} \VariableTok{True}
            \ControlFlowTok{if}\NormalTok{ carry }\OperatorTok{==} \DecValTok{0}\NormalTok{:}
                \ControlFlowTok{for}\NormalTok{ j }\KeywordTok{in} \BuiltInTok{range}\NormalTok{(len_rhs }\OperatorTok{-} \DecValTok{1}\NormalTok{, }\DecValTok{-1}\NormalTok{, }\DecValTok{-1}\NormalTok{):  }\CommentTok{# O(k)}
                    \ControlFlowTok{if}\NormalTok{ res_rem.b[i }\OperatorTok{+}\NormalTok{ j] }\OperatorTok{>}\NormalTok{ rhs.b[j]:}
\NormalTok{                        ok }\OperatorTok{=} \VariableTok{True}
                        \ControlFlowTok{break}
                    \ControlFlowTok{elif}\NormalTok{ res_rem.b[i }\OperatorTok{+}\NormalTok{ j] }\OperatorTok{<}\NormalTok{ rhs.b[j]:}
\NormalTok{                        ok }\OperatorTok{=} \VariableTok{False}
                        \ControlFlowTok{break}
  
            \ControlFlowTok{if}\NormalTok{ ok:}
\NormalTok{                res_div.b[i] }\OperatorTok{=} \DecValTok{1}
                \ControlFlowTok{for}\NormalTok{ j }\KeywordTok{in} \BuiltInTok{range}\NormalTok{(}\DecValTok{0}\NormalTok{, len_rhs):  }\CommentTok{# O(k)}
\NormalTok{                    res_rem.b[i }\OperatorTok{+}\NormalTok{ j] }\OperatorTok{-=}\NormalTok{ rhs.b[j]}
                    \ControlFlowTok{if}\NormalTok{ res_rem.b[i }\OperatorTok{+}\NormalTok{ j] }\OperatorTok{<} \DecValTok{0}\NormalTok{:}
\NormalTok{                        res_rem.b[i }\OperatorTok{+}\NormalTok{ j] }\OperatorTok{+=} \DecValTok{2}
\NormalTok{                        res_rem.b[i }\OperatorTok{+}\NormalTok{ j }\OperatorTok{+} \DecValTok{1}\NormalTok{] }\OperatorTok{-=} \DecValTok{1}
            \ControlFlowTok{else}\NormalTok{:}
\NormalTok{                res_div.b[i] }\OperatorTok{=} \DecValTok{0}
\NormalTok{            carry }\OperatorTok{=}\NormalTok{ res_rem.b[i }\OperatorTok{+}\NormalTok{ len_rhs }\OperatorTok{-} \DecValTok{1}\NormalTok{]}
  
        \ControlFlowTok{while}\NormalTok{ res_div.b }\KeywordTok{and}\NormalTok{ res_div.b[}\OperatorTok{-}\DecValTok{1}\NormalTok{] }\OperatorTok{==} \DecValTok{0}\NormalTok{:}
\NormalTok{            res_div.b.pop()}
  
        \ControlFlowTok{while}\NormalTok{ res_rem.b }\KeywordTok{and}\NormalTok{ res_rem.b[}\OperatorTok{-}\DecValTok{1}\NormalTok{] }\OperatorTok{==} \DecValTok{0}\NormalTok{:}
\NormalTok{            res_rem.b.pop()}
  
        \ControlFlowTok{return}\NormalTok{ res_div, res_rem}
  
    \KeywordTok{def}\NormalTok{ div(}\VariableTok{self}\NormalTok{, rhs):}
        \ControlFlowTok{if} \KeywordTok{not} \BuiltInTok{isinstance}\NormalTok{(rhs, Binary):}
            \ControlFlowTok{raise} \PreprocessorTok{TypeError}\NormalTok{(}\StringTok{"rhs is not Binary"}\NormalTok{)}
  
        \ControlFlowTok{return} \VariableTok{self}\NormalTok{._div_rem(rhs)[}\DecValTok{0}\NormalTok{]}
  
    \KeywordTok{def}\NormalTok{ rem(}\VariableTok{self}\NormalTok{, rhs):}
        \ControlFlowTok{if} \KeywordTok{not} \BuiltInTok{isinstance}\NormalTok{(rhs, Binary):}
            \ControlFlowTok{raise} \PreprocessorTok{TypeError}\NormalTok{(}\StringTok{"rhs is not Binary"}\NormalTok{)}
  
        \ControlFlowTok{return} \VariableTok{self}\NormalTok{._div_rem(rhs)[}\DecValTok{1}\NormalTok{]}
  
    \KeywordTok{def}\NormalTok{ greaterthan(}\VariableTok{self}\NormalTok{, rhs):}
        \ControlFlowTok{if} \KeywordTok{not} \BuiltInTok{isinstance}\NormalTok{(rhs, Binary):}
            \ControlFlowTok{raise} \PreprocessorTok{TypeError}\NormalTok{(}\StringTok{"rhs is not Binary"}\NormalTok{)}
  
\NormalTok{        len_lhs, len_rhs }\OperatorTok{=} \VariableTok{self}\NormalTok{.}\BuiltInTok{len}\NormalTok{(), rhs.}\BuiltInTok{len}\NormalTok{()}
  
        \ControlFlowTok{if}\NormalTok{ len_lhs }\OperatorTok{!=}\NormalTok{ len_rhs:}
            \ControlFlowTok{return}\NormalTok{ len_lhs }\OperatorTok{>}\NormalTok{ len_rhs}
  
        \ControlFlowTok{for}\NormalTok{ i }\KeywordTok{in} \BuiltInTok{range}\NormalTok{(len_rhs }\OperatorTok{-} \DecValTok{1}\NormalTok{, }\DecValTok{-1}\NormalTok{, }\DecValTok{-1}\NormalTok{):}
            \ControlFlowTok{if} \VariableTok{self}\NormalTok{.b[i] }\OperatorTok{!=}\NormalTok{ rhs.b[i]:}
                \ControlFlowTok{return} \VariableTok{self}\NormalTok{.b[i] }\OperatorTok{>}\NormalTok{ rhs.b[i]}
  
        \ControlFlowTok{return} \VariableTok{False}
  
    \KeywordTok{def}\NormalTok{ equals(}\VariableTok{self}\NormalTok{, rhs):}
        \ControlFlowTok{if} \KeywordTok{not} \BuiltInTok{isinstance}\NormalTok{(rhs, Binary):}
            \ControlFlowTok{raise} \PreprocessorTok{TypeError}\NormalTok{(}\StringTok{"rhs is not Binary"}\NormalTok{)}
  
\NormalTok{        len_lhs, len_rhs }\OperatorTok{=} \VariableTok{self}\NormalTok{.}\BuiltInTok{len}\NormalTok{(), rhs.}\BuiltInTok{len}\NormalTok{()}
  
        \ControlFlowTok{if}\NormalTok{ len_lhs }\OperatorTok{!=}\NormalTok{ len_rhs:}
            \ControlFlowTok{return} \VariableTok{False}
  
        \ControlFlowTok{for}\NormalTok{ i }\KeywordTok{in} \BuiltInTok{range}\NormalTok{(len_rhs):}
            \ControlFlowTok{if} \VariableTok{self}\NormalTok{.b[i] }\OperatorTok{!=}\NormalTok{ rhs.b[i]:}
                \ControlFlowTok{return} \VariableTok{False}
  
        \ControlFlowTok{return} \VariableTok{True}
  
  
\KeywordTok{def}\NormalTok{ euclid(a, b):}
\NormalTok{    a, b }\OperatorTok{=}\NormalTok{ Binary(a), Binary(b)}
    \ControlFlowTok{while}\NormalTok{ b.greaterthan(Binary(}\DecValTok{0}\NormalTok{)):}
\NormalTok{        r }\OperatorTok{=}\NormalTok{ a.rem(b)  }\CommentTok{# so a = bu + r}
        \ControlFlowTok{if}\NormalTok{ r.equals(Binary(}\DecValTok{0}\NormalTok{)):}
            \ControlFlowTok{return}\NormalTok{ b}
\NormalTok{        s }\OperatorTok{=}\NormalTok{ b.rem(r)  }\CommentTok{# so b = rv + s}
\NormalTok{        a }\OperatorTok{=}\NormalTok{ r}
\NormalTok{        b }\OperatorTok{=}\NormalTok{ s}
    \ControlFlowTok{return}\NormalTok{ a}
  
  
\BuiltInTok{print}\NormalTok{(euclid(}\DecValTok{12}\NormalTok{, }\DecValTok{8}\NormalTok{))}
  
\end{Highlighting}
\end{Shaded}

\begin{Shaded}
\begin{Highlighting}[]
\NormalTok{4}
\end{Highlighting}
\end{Shaded}

\end{document}
