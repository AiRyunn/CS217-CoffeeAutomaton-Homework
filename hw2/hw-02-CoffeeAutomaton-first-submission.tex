% Options for packages loaded elsewhere
\PassOptionsToPackage{unicode}{hyperref}
\PassOptionsToPackage{hyphens}{url}


\documentclass[
]{article}

%\usepackage{tcolorbox}
%\newtcolorbox{myquote}{colback=red!5!white, colframe=red!75!black}
%\newtcolorbox{myShaded}{colback=red!5!white, colframe=red!75!black}
% redefine the 'quote' environment to use this 'myquote' environment
%\renewenvironment{quote}{\begin{myquote}}{\end{myquote}}
%\newenvironment{Highlighting}{\begin{myquote}}{\end{myquote}}

\usepackage{lmodern}
\usepackage{amssymb,amsmath}


\usepackage{ifxetex,ifluatex}
\ifnum 0\ifxetex 1\fi\ifluatex 1\fi=0 % if pdftex
  \usepackage{mathpazo} % Palatino-like math fonts

  \usepackage[T1]{fontenc}
  \usepackage[utf8]{inputenc}
  \usepackage{textcomp} % provide euro and other symbols
\else % if luatex or xetex

% Load mathpazo as a math font
\usepackage{mathpazo}

% Load Pagella as a text font by specifying no-math to fontspec
%\usepackage[no-math]{fontspec}
%\setmainfont[Numbers=Proportional]{TeX Gyre Pagella}

  \usepackage{unicode-math}
  \defaultfontfeatures{Scale=MatchLowercase}
  \defaultfontfeatures[\rmfamily]{Ligatures=TeX,Scale=1}
  \setmainfont[Numbers = Proportional]{TeX Gyre Pagella}
  \setmonofont[]{Ubuntu Mono}
\fi
% Use upquote if available, for straight quotes in verbatim environments
\IfFileExists{upquote.sty}{\usepackage{upquote}}{}
\IfFileExists{microtype.sty}{% use microtype if available
  \usepackage[]{microtype}
  \UseMicrotypeSet[protrusion]{basicmath} % disable protrusion for tt fonts
}{}
\makeatletter
\@ifundefined{KOMAClassName}{% if non-KOMA class
  \IfFileExists{parskip.sty}{%
    \usepackage{parskip}
  }{% else
    \setlength{\parindent}{0pt}
    \setlength{\parskip}{6pt plus 2pt minus 1pt}}
}{% if KOMA class
  \KOMAoptions{parskip=half}}
\makeatother
\usepackage{xcolor}
\IfFileExists{xurl.sty}{\usepackage{xurl}}{} % add URL line breaks if available
\IfFileExists{bookmark.sty}{\usepackage{bookmark}}{\usepackage{hyperref}}
\hypersetup{
  pdftitle={Homework \#2},
  pdfauthor={Coffee Automaton},
  hidelinks,
  pdfcreator={LaTeX via pandoc}}
\urlstyle{same} % disable monospaced font for URLs
\usepackage[margin = 1in]{geometry}
\usepackage{color}
\usepackage{fancyvrb}
\newcommand{\VerbBar}{|}
\newcommand{\VERB}{\Verb[commandchars=\\\{\}]}
\DefineVerbatimEnvironment{Highlighting}{Verbatim}{commandchars=\\\{\}}
% Add ',fontsize=\small' for more characters per line
\usepackage{framed}
\definecolor{shadecolor}{RGB}{248,248,248}
\newenvironment{Shaded}{\begin{snugshade}}{\end{snugshade}}
\newcommand{\AlertTok}[1]{\textcolor[rgb]{0.94,0.16,0.16}{#1}}
\newcommand{\AnnotationTok}[1]{\textcolor[rgb]{0.56,0.35,0.01}{\textbf{\textit{#1}}}}
\newcommand{\AttributeTok}[1]{\textcolor[rgb]{0.77,0.63,0.00}{#1}}
\newcommand{\BaseNTok}[1]{\textcolor[rgb]{0.00,0.00,0.81}{#1}}
\newcommand{\BuiltInTok}[1]{#1}
\newcommand{\CharTok}[1]{\textcolor[rgb]{0.31,0.60,0.02}{#1}}
\newcommand{\CommentTok}[1]{\textcolor[rgb]{0.56,0.35,0.01}{\textit{#1}}}
\newcommand{\CommentVarTok}[1]{\textcolor[rgb]{0.56,0.35,0.01}{\textbf{\textit{#1}}}}
\newcommand{\ConstantTok}[1]{\textcolor[rgb]{0.00,0.00,0.00}{#1}}
\newcommand{\ControlFlowTok}[1]{\textcolor[rgb]{0.13,0.29,0.53}{\textbf{#1}}}
\newcommand{\DataTypeTok}[1]{\textcolor[rgb]{0.13,0.29,0.53}{#1}}
\newcommand{\DecValTok}[1]{\textcolor[rgb]{0.00,0.00,0.81}{#1}}
\newcommand{\DocumentationTok}[1]{\textcolor[rgb]{0.56,0.35,0.01}{\textbf{\textit{#1}}}}
\newcommand{\ErrorTok}[1]{\textcolor[rgb]{0.64,0.00,0.00}{\textbf{#1}}}
\newcommand{\ExtensionTok}[1]{#1}
\newcommand{\FloatTok}[1]{\textcolor[rgb]{0.00,0.00,0.81}{#1}}
\newcommand{\FunctionTok}[1]{\textcolor[rgb]{0.00,0.00,0.00}{#1}}
\newcommand{\ImportTok}[1]{#1}
\newcommand{\InformationTok}[1]{\textcolor[rgb]{0.56,0.35,0.01}{\textbf{\textit{#1}}}}
\newcommand{\KeywordTok}[1]{\textcolor[rgb]{0.13,0.29,0.53}{\textbf{#1}}}
\newcommand{\NormalTok}[1]{#1}
\newcommand{\OperatorTok}[1]{\textcolor[rgb]{0.81,0.36,0.00}{\textbf{#1}}}
\newcommand{\OtherTok}[1]{\textcolor[rgb]{0.56,0.35,0.01}{#1}}
\newcommand{\PreprocessorTok}[1]{\textcolor[rgb]{0.56,0.35,0.01}{\textit{#1}}}
\newcommand{\RegionMarkerTok}[1]{#1}
\newcommand{\SpecialCharTok}[1]{\textcolor[rgb]{0.00,0.00,0.00}{#1}}
\newcommand{\SpecialStringTok}[1]{\textcolor[rgb]{0.31,0.60,0.02}{#1}}
\newcommand{\StringTok}[1]{\textcolor[rgb]{0.31,0.60,0.02}{#1}}
\newcommand{\VariableTok}[1]{\textcolor[rgb]{0.00,0.00,0.00}{#1}}
\newcommand{\VerbatimStringTok}[1]{\textcolor[rgb]{0.31,0.60,0.02}{#1}}
\newcommand{\WarningTok}[1]{\textcolor[rgb]{0.56,0.35,0.01}{\textbf{\textit{#1}}}}
\setlength{\emergencystretch}{3em} % prevent overfull lines
\providecommand{\tightlist}{%
  \setlength{\itemsep}{0pt}\setlength{\parskip}{0pt}}
\setcounter{secnumdepth}{-\maxdimen} % remove section numbering

% text wrapping
\usepackage{fvextra}
\DefineVerbatimEnvironment{Highlighting}{Verbatim}{breaklines,commandchars=\\\{\}}
\renewenvironment{verbatim}{\begin{Shaded}}{\end{Shaded}}


\title{
  \vspace{2in}
  \textmd{\textbf{Homework \#2}}
  \normalsize\vspace{0.1in}\\
  \textmd{\textbf{CS 217 @ SJTU}}
  \normalsize\vspace{0.1in}\\
}

\author{Coffee Automaton}
\date{}

\begin{document}

\noindent
\large\textbf{Homework \#2}
\hfill
\textbf{Coffee Automaton} \\
\normalsize {\bf CS 217 @ SJTU} \hfill ACM Class, Zhiyuan College, SJTU\\
Prof.~{\bf Dominik Scheder} \hfill Due Date: September 29, 2019\\
  TA.~{\bf Tang Shuyang}
\hfill Submit Date: \today


\hypertarget{exercise-1}{%
\section{Exercise 1}\label{exercise-1}}

The algorithm is described as follows.

\begin{enumerate}
\def\labelenumi{\arabic{enumi}.}
\item
  Make the array into \(\frac{n}{2}\) pairs.
\item
  Sort each of the pair (makes \(\frac{n}{2}\) comparisons).
\item
  For each of the pair, take the smaller one to the S group, and the
  larger one to the L group.
\item
  Find the smallest in the S group and the largest in the L group as the
  answer (each makes \(\frac{n}{2}-1\) comparisons).
\end{enumerate}

The algorithm totally makes \(\frac{3}{2}n-2\) comparisons.

Here's the code implemented for the algorithm.

\begin{Shaded}
\begin{Highlighting}[]
\ImportTok{from}\NormalTok{ random }\ImportTok{import}\NormalTok{ shuffle}
  
\NormalTok{count }\OperatorTok{=} \DecValTok{0}
  
  
\KeywordTok{def}\NormalTok{ lessthan(a, b):}
    \KeywordTok{global}\NormalTok{ count}
\NormalTok{    count }\OperatorTok{+=} \DecValTok{1}
    \ControlFlowTok{return}\NormalTok{ a }\OperatorTok{<}\NormalTok{ b}
  
  
\KeywordTok{def}\NormalTok{ greaterthan(a, b):}
    \KeywordTok{global}\NormalTok{ count}
\NormalTok{    count }\OperatorTok{+=} \DecValTok{1}
    \ControlFlowTok{return}\NormalTok{ a }\OperatorTok{>}\NormalTok{ b}
  
  
\KeywordTok{def}\NormalTok{ find_min_and_max(array):}
\NormalTok{    n }\OperatorTok{=} \BuiltInTok{len}\NormalTok{(array)}
\NormalTok{    mi, mx }\OperatorTok{=} \VariableTok{None}\NormalTok{, }\VariableTok{None}
    \ControlFlowTok{for}\NormalTok{ i }\KeywordTok{in} \BuiltInTok{range}\NormalTok{(}\DecValTok{0}\NormalTok{, n, }\DecValTok{2}\NormalTok{):}
        \ControlFlowTok{if} \KeywordTok{not}\NormalTok{ lessthan(array[i], array[i }\OperatorTok{+} \DecValTok{1}\NormalTok{]):}
            \CommentTok{# swap the pairs}
\NormalTok{            array[i], array[i }\OperatorTok{+} \DecValTok{1}\NormalTok{] }\OperatorTok{=}\NormalTok{ array[i }\OperatorTok{+} \DecValTok{1}\NormalTok{], array[i]}
    \ControlFlowTok{for}\NormalTok{ i }\KeywordTok{in} \BuiltInTok{range}\NormalTok{(}\DecValTok{0}\NormalTok{, n, }\DecValTok{2}\NormalTok{):}
        \ControlFlowTok{if}\NormalTok{ mi }\KeywordTok{is} \VariableTok{None} \KeywordTok{or}\NormalTok{ lessthan(array[i], mi):}
\NormalTok{            mi }\OperatorTok{=}\NormalTok{ array[i]}
        \ControlFlowTok{if}\NormalTok{ mx }\KeywordTok{is} \VariableTok{None} \KeywordTok{or}\NormalTok{ greaterthan(array[i }\OperatorTok{+} \DecValTok{1}\NormalTok{], mi):}
\NormalTok{            mx }\OperatorTok{=}\NormalTok{ array[i }\OperatorTok{+} \DecValTok{1}\NormalTok{]}
    \ControlFlowTok{return}\NormalTok{ mi, mx}
  
  
\NormalTok{n }\OperatorTok{=} \DecValTok{20}
\NormalTok{array }\OperatorTok{=}\NormalTok{ [i }\ControlFlowTok{for}\NormalTok{ i }\KeywordTok{in} \BuiltInTok{range}\NormalTok{(n)]}
\NormalTok{shuffle(array)}
  
\BuiltInTok{print}\NormalTok{(}\SpecialStringTok{f"array: }\SpecialCharTok{\{}\NormalTok{array}\SpecialCharTok{\}}\SpecialStringTok{"}\NormalTok{)}
  
\BuiltInTok{print}\NormalTok{(}\SpecialStringTok{f"min and max: }\SpecialCharTok{\{}\NormalTok{find_min_and_max(array)}\SpecialCharTok{\}}\SpecialStringTok{"}\NormalTok{)}
\BuiltInTok{print}\NormalTok{(}\SpecialStringTok{f"3 n / 2 - 2 = }\SpecialCharTok{\{}\DecValTok{3} \OperatorTok{*} \SpecialCharTok{n} \OperatorTok{//} \DecValTok{2} \OperatorTok{-} \DecValTok{2}\SpecialCharTok{\}}\SpecialStringTok{"}\NormalTok{)}
\BuiltInTok{print}\NormalTok{(}\SpecialStringTok{f"number of comparisons: }\SpecialCharTok{\{}\NormalTok{count}\SpecialCharTok{\}}\SpecialStringTok{"}\NormalTok{)}
  
\end{Highlighting}
\end{Shaded}

\begin{Shaded}
\begin{Highlighting}[]
\NormalTok{array: [17, 5, 10, 2, 11, 13, 16, 7, 14, 18, 12, 3, 1, 6, 4, 15, 0, 9, 8, 19]}
\NormalTok{min and max: (0, 19)}
\NormalTok{3 n / 2 - 2 = 28}
\NormalTok{number of comparisons: 28}
\end{Highlighting}
\end{Shaded}

\hypertarget{exercise-2}{%
\section{Exercise 2}\label{exercise-2}}

\href{https://www.zhihu.com/question/33113457}{solution}

\hypertarget{exercise-3}{%
\section{Exercise 3}\label{exercise-3}}

\href{http://theory.stanford.edu/~tim/w11/l/qsort.pdf}{solution}

\hypertarget{exercise-4}{%
\section{Exercise 4}\label{exercise-4}}

\hypertarget{exercise-5}{%
\section{Exercise 5}\label{exercise-5}}

\hypertarget{exercise-6}{%
\section{Exercise 6}\label{exercise-6}}

\hypertarget{exercise-7}{%
\section{Exercise 7}\label{exercise-7}}

\end{document}
