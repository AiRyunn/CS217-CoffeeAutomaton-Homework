\documentclass{article}
\usepackage[utf8]{inputenc}

\usepackage{caption}
\usepackage{graphicx, subfig}
\usepackage{subfigure}
\usepackage{amsmath}
\usepackage{mathrsfs}
\usepackage{amsfonts}
\usepackage{natbib}
\usepackage{graphicx}
\usepackage{amssymb}
\usepackage{upgreek}
\usepackage{listings}
\usepackage{xcolor}
\usepackage{dsfont}


\title{Homework #5}
\author{Coffee Automaton}
\date{October 2019}

\begin{document}

\maketitle

\textbf{Exercise 1}

If there is not such a flow of value $k$, then there must be a cut S that V\\S of $c(S) < k$. Now consider where $r$ is.

If $r \in S$, then S, V separates $r$ and $t$. But there is a flow from $r$ to $t$ of value k, and flow is smaller than cut, so it's a contradiction.

If $r \in V\\S$ , similarly, separates $s$ and $r$ but there is a flow from $s$ to $r$ of value k, so it's a contradiction.

Hence there is a flow of value $k$.



\textbf{Exercise 3}

A premise: there is no path that from $s$ and $t$ with no intermediate point.

First prove that not both cases occur.

If there are $k$ vertex disjoint paths $p_1 , ...,  p_k$ from $s$ to $t$, $k-1$ vertices can at most cut $k-1$ paths of them and at least 1 paths left. So we can't find vertices $v_1, ...v_{k-1}$ that $G - \{v_1, ...v_{k-1}\}$ contains no $s-t$ path.

If there are $k-1$ vertices $v_1, ...v_{k-1} \in V \\ \{ s,t \}$, then we at most recover $k-1$ vertex disjoint paths(but there may contain $k$ or more paths, but they must share some same vertices). So we can't find $k$ vertex disjoint paths from $s$ to $t$.

Then, we prove that it's impossible both cases not happen.

Assume that both not happen, it means that:

1.There are $x$ vertex disjoint paths $p_1, ... , p_x$ from $s$ to $t$ with $x < k$(x can be zero).

2.There are $y$ vertices $v_1,... v_y$ that $G - \{v_1, ...v_{y-1}\}$ contains not $s-t$ path with $y > k$.(we can remove all the vertices except $s$ and $t$)
0
Now for $x$ vertex disjoint paths, every path extract a vertex expect $s$ and $t$, and we get vertices $v_x1, ... , v_xx$ and they must differ from one another because they come from vertex disjoint paths. Now we remove them, and we get graph $G - \{v_x1, ...v_{xx}\}$ that has no vertex disjoint path, which means has no $s-t$ paths. Then we find a way to satisfy condition 2 with $x$ vertices, so it must be true that $x \geqslant y$. But it's against with the premise $x < k < y$, so contradiction. One of them must occur.



\textbf{Exercise 4}

First do some observation:

1. For the vertex v, if it has $i$ 1 in its number, then it belongs $L_i$. So $L_i$ has $C_{n}^{i}$ vertices.

2. There is an edge between $x_i$ and $x_{i+1}$ if and only if there is one difference in their numbers. So there is $n-i$ edges.

3. There is no edge in $L_{i}$.

Now consider a network $G = \{ V, E, c \}$ that:

$V = L_{i} \cup L_{i+1} \cup \{s, t\}$

$E = \{  (u,v) | u \in L_{i}, v \in L_{i+1} \} \cup \{(s,u) | u \in L_{i} \} \cup \{(v,t) | v \in L_{i+1} \}$

$$c(u, v)=
\begin{cases}
+\infty & u\in L_{i}, v \in  L_{i+1}, (u,v) \in E\\
1& u=s, v \in  L_{i}\\
1& u \in  L_{i+1} , v=t   \\
0& otherwise\\
\end{cases}$$

And define $f(u,v)$:

$$f(u, v)=
\begin{cases}
\frac{1}{d_1} & u\in L_{i}, v \in L_{i+1}, (u,v) \in E\\
-\frac{1}{d_1} & u\in L_{i+1}, v \in L_{i}, (v,u) \in E\\
1 & u=s, v \in L_{i}\\
-1 & u\in L_{i}, v = s\\
\frac{d_2}{d_1} & u\in L_{i+1}, v = t\\
-\frac{d_2}{d_1} & u=t, v\in L_{i+1}\\
0 & otherwise\\
\end{cases}$$

And we define $|V_1|d_1 = |V_2|d_2, so \frac{d_2}{d_1} = \frac{i+1}{n-i-1}$.

Then we have $f(u,v) = - f(v,u), f(u,v) \leqslant c(u,v)$, and $\sum f(u,v) = 1- d_1 * \frac{1}{d_1} = 0$ or $\sum f(u,v) = \frac{d_2}{d_1} - d_2 *\frac{1}{d_1} = 0$ for $u\in L_{i}$

So we get a flow $f$ with $val(f) = |L_{i}|$.

Now let $S = \{ s\}$, then S, V\\S is a cut and $c(S,V\\S) = |V_1| = val(f)$, so $f$ is a maximum flow, and there is an integral maximal flow of value $L_{i}$. 

So there is a matching in $L_{i} \cup L_{i+1}$ of size $L_{i}$.



\textbf{Exercise 5}

We define $s$ and $t$, and define $G(V,E)$ that:

$V = L_{i} \cup L_{i+1} \cup ... \cup L_{\frac{n}{2}} \cup ... \cup L_{n-i}  \cup \{s, t\}$

$E = \{  (u,v) | u \in L_{i}, v \in L_{i+1} \}\cup ... \cup \{(s,u) | u \in L_{i} \} \cup \{(v,t) | v \in L_{n-i} \}$

We try to prove it by Menger’s Theorem. That is, if we can't find $C^{i}_{n} - 1$ vertices that cut the G, then there must be $C^{i}_{n}$ vertex disjoint paths from $s$ to $t$.

Denote $C_{i}^{n}$ as $m_i$. First consider the $L_{\frac{n}{2}-1} - (L_{\frac{n}{2}}) - L_{\frac{n}{2}+1}$. If there is not $L_{\frac{n}{2}}$, then there is $m_{\frac{n}{2}-1}$ edges between $s$ and $L_{\frac{n}{2} - 1}$, $L_{\frac{n}{2} - 1}$ and $L_{\frac{n}{2} + 1}$, $L_{\frac{n}{2} + 1}$ and $t$. So there is $m_{\frac{n}{2}-1}$ vertex disjoint paths. If there is $L_{\frac{n}{2}}$, the $L_{\frac{n}{2} -1}$ paths covers $L_{\frac{n}{2} - 1}$ vertices, but we can at most choose $L_{\frac{n}{2} - 1} - 1$ vertices. So we can't find such vertices to get a no s-t path graph.

Then we can do this induction from $\frac{n}{2}-1$ to $0$, and get the conclusion. 



\textbf{Exercise 6}

Because $v(G)$ is the size of a maximum matching of bipartite graph G, so there is at least $2v(G)$ vertices and $v(G) edges$, and the minimum cover size is $v(G)$, which is same as $v(G)$ parallel edges, we can choose one of two endpoints at every edge, so this graph has obviously $2^{v(G)}$ minimum vertex covers. If we add additional edges, then we add extra restrictions and we may not choose vertices optionally, and no more new covers because one of two endpoints of one edge must be chosen. So at most we have $2^{v(G)}$ minmum covers.

\textbf{Exercise 7}

Denote that this graph have $n$ vertices and $m$ edges.

The vertices maybe $2k$ or $2k+1$.

So the answer may $3^k$.




\end{document}
